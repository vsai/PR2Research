%This work is licensed under the
%Creative Commons Attribution-Share Alike 3.0 United States License.
%To view a copy of this license, visit
%http://creativecommons.org/licenses/by-sa/3.0/us/ or send a letter to
%Creative Commons,
%171 Second Street, Suite 300,
%San Francisco, California, 94105, USA.

\section{Shorten Encoding}
As with decoding, one needs \texttt{unsigned}, \texttt{signed} and \texttt{long}
functions:

\subsubsection{Writing \texttt{unsigned}}
\ALGORITHM{a bit count $c$, an unsigned value}{a written \texttt{unsigned} value}
\SetKwData{VALUE}{value}
\SetKwData{MSB}{MSB}
\SetKwData{LSB}{LSB}
$\text{\MSB} \leftarrow \lfloor\text{\VALUE} \div 2 ^ c\rfloor$\;
$\text{\LSB} \leftarrow \text{\VALUE} - \text{\MSB} \times 2 ^ c$\;
$\MSB \rightarrow$ \WUNARY with stop bit 1\;
$\LSB \rightarrow$ \WRITE $c$ unsigned bits\;
\EALGORITHM

\subsubsection{Writing \texttt{signed}}
\ALGORITHM{a bit count $c$, a signed value}{a written \texttt{signed} value}
\SetKwData{VALUE}{value}
\eIf{$\text{\VALUE} \geq 0$}{
  write $\texttt{unsigned}(c + 1~,~\text{\VALUE} \times 2)$\;
}{
  write $\texttt{unsigned}(c + 1~,~(-\text{\VALUE} - 1) \times 2 + 1)$\;
}
\EALGORITHM

\subsubsection{Writing \texttt{long}}
\ALGORITHM{an unsigned value}{a written \texttt{long} value}
\SetKwData{VALUE}{value}
\eIf{$\text{\VALUE} = 0$}{
  write $\texttt{unsigned}(2~,~0)$\;
  write $\texttt{unsigned}(0~,~0)$\;
}{
  $\text{LSBs} \leftarrow \lfloor\log_2(\text{\VALUE})\rfloor + 1$\;
  write $\texttt{unsigned}(2~,~\text{LSBs})$\;
  write $\texttt{unsigned}(\text{LSBs}~,~\text{\VALUE})$\;
}
\EALGORITHM

\clearpage

{\relsize{-1}
  \ALGORITHM{PCM frames, a block size parameter, a wave or aiff header and footer}{an encoded Shorten file}
  \SetKwData{BITSPERSAMPLE}{bits per sample}
  \SetKwData{CHANNELS}{channel count}
  \SetKwData{BLOCKSIZE}{block size}
  \SetKw{IN}{in}
  \SetKwData{LEFTSHIFT}{left shift}
  \SetKwData{WASTEDBITS}{wasted BPS}
  \SetKwData{SHIFTED}{shifted}
  \SetKwData{DIFF}{diff}
  \SetKwData{RESIDUALS}{residual}
  \SetKwData{ENERGY}{energy}
  \SetKwData{SAMPLES}{samples}
  \SetKwData{CHANNEL}{channel}
  \hyperref[shorten:write_header]{write Shorten header with \BITSPERSAMPLE, \CHANNELS and \BLOCKSIZE}\;
  write \texttt{unsigned}(2~,~9)\tcc*[r]{VERBATIM command}
  write \texttt{unsigned}(5~,~header byte count)\;
  \ForEach{byte \IN header}{
    write \texttt{unsigned}(8~,~\textit{byte})\;
  }
  $\text{\LEFTSHIFT} \leftarrow 0$\;
  \BlankLine
  \While{PCM frames remain}{
    $\text{\SAMPLES} \leftarrow$ take \BLOCKSIZE PCM frames from input stream\;
    \If{$\text{\SAMPLES PCM frame count} \neq \BLOCKSIZE$}{
      $\text{\BLOCKSIZE} \leftarrow \textit{\SAMPLES PCM frame count}$\;
      write \texttt{unsigned}(2~,~5)\tcc*[r]{BLOCKSIZE command}
      write \texttt{long}(\BLOCKSIZE)\;
    }
    \ForEach{\CHANNEL \IN \SAMPLES}{
      \eIf{$\text{all samples in \CHANNEL} = 0$}{
        write \texttt{unsigned}(2~,~8)\tcc*[r]{ZERO command}
        \hyperref[shorten:wrap_samples]{wrap \CHANNEL for next set of channel data}\;
      }{
        $\text{\WASTEDBITS} \leftarrow$ \hyperref[shorten:calculate_wasted_bps]{calculate wasted BPS for \CHANNEL}\;
        \If{$\text{\LEFTSHIFT} \neq \text{\WASTEDBITS}$}{
          $\text{\LEFTSHIFT} \leftarrow \text{\WASTEDBITS}$\;
          write \texttt{unsigned}(2~,~6)\tcc*[r]{BITSHIFT command}
          write \texttt{unsigned}(2~,~\LEFTSHIFT)\;
        }
        \For{$i \leftarrow 0$ \emph{\KwTo}\BLOCKSIZE}{
          $\text{\SHIFTED}_i \leftarrow \text{\CHANNEL}_i \div 2 ^ {\text{\LEFTSHIFT}}$\;
        }
        $\left.\begin{tabular}{r}
          \DIFF \\
          \ENERGY \\
          \RESIDUALS \\
        \end{tabular}\right\rbrace \leftarrow$ \hyperref[shorten:compute_best_diff]{compute best \texttt{DIFF}, energy and residual values for \SHIFTED}\;
        write \texttt{unsigned}(2~,~\DIFF)\tcc*[r]{DIFF command}
        write \texttt{unsigned}(3~,~\ENERGY)\;
        \ForEach{r \IN \RESIDUALS}{
          write \texttt{signed}(\ENERGY~,~r)\;
        }
        \hyperref[shorten:wrap_samples]{wrap \SHIFTED for next set of channel data}\;
      }
    }
  }
  \BlankLine
  \If{$\text{footer byte count} > 0$}{
    write \texttt{unsigned}(2~,~9)\tcc*[r]{VERBATIM command}
    write \texttt{unsigned}(5~,~footer byte count)\;
    \ForEach{byte \IN footer}{
      write \texttt{unsigned}(8~,~\textit{byte})\;
    }
  }
  \BlankLine
  write \texttt{unsigned}(2~,~5)\tcc*[r]{QUIT command}
  \BlankLine
  \tcc{Shorten output (not including 5 bytes of magic + version)
  must be a multiple of 4 bytes, or the reference decoder's
  bit stream reader will fail}
  byte align the stream\;
  \While{$(\text{total file size} - 5) \bmod 4 = 0$}{
    \WRITE 0 in 8 unsigned bits\;
  }
\EALGORITHM
}

\clearpage

\subsection{Writing Shorten Header}
\label{shorten:write_header}
\ALGORITHM{the input stream's bits-per-sample, sample signedness and endianness;\newline channel count and initial block size}{a Shorten header}
\SetKwData{BITSPERSAMPLE}{bits per sample}
\SetKwData{ENDIANNESS}{endianness}
\SetKwData{SIGNEDNESS}{signedness}
\SetKwData{CHANNELS}{channel count}
\SetKwData{BLOCKSIZE}{block size}
$\texttt{"ajkg"} \rightarrow$ \WRITE 4 bytes\;
$2 \rightarrow$ \WRITE 8 unsigned bits\;
\uIf{$\BITSPERSAMPLE = 8$}{
  \eIf{$\SIGNEDNESS = signed$}{
    write \texttt{long}(1)\tcc*[r]{signed, 8 bit}
  }{
    write \texttt{long}(2)\tcc*[r]{unsigned, 8 bit}
  }
}
\uElseIf{$\BITSPERSAMPLE = 16$}{
  \eIf{$\SIGNEDNESS = signed$}{
    \eIf{$\ENDIANNESS = big$}{
      write \texttt{long}(3)\tcc*[r]{signed, 16 bit, big-endian}
    }{
      write \texttt{long}(5)\tcc*[r]{signed, 16 bit, little-endian}
    }
  }{
    \eIf{$\ENDIANNESS = big$}{
      write \texttt{long}(4)\tcc*[r]{unsigned, 16 bit, big-endian}
    }{
      write \texttt{long}(6)\tcc*[r]{unsigned, 16 bit, little-endian}
    }
  }

}
\Else{
  unsupported number of bits per sample\;
}
write \texttt{long}(\CHANNELS)\;
write \texttt{long}(\BLOCKSIZE)\;
write \texttt{long}(0)\tcc*[r]{maximum LPC}
write \texttt{long}(0)\tcc*[r]{mean count}
write \texttt{long}(0)\tcc*[r]{bytes to skip}
\EALGORITHM

\clearpage

\subsection{Calculating Wasted Bits per Sample}
\label{shorten:calculate_wasted_bps}
\ALGORITHM{a list of signed PCM samples}{an unsigned integer}
\SetKwData{WASTEDBPS}{wasted bps}
\SetKwData{SAMPLE}{sample}
\SetKwFunction{MIN}{min}
\SetKwFunction{WASTED}{wasted}
$\text{\WASTEDBPS} \leftarrow \infty$\tcc*[r]{maximum unsigned integer}
\For{$i \leftarrow 0$ \emph{\KwTo}sample count}{
  $\text{\WASTEDBPS} \leftarrow \MIN(\WASTED(\text{\SAMPLE}_i)~,~\text{\WASTEDBPS})$\;
}
\eIf(\tcc*[f]{all samples are 0}){$\WASTEDBPS = \infty$}{
  \Return 0\;
}{
  \Return \WASTEDBPS\;
}
\EALGORITHM
where the \texttt{wasted} function is defined as:
\begin{equation*}
  \texttt{wasted}(x) =
  \begin{cases}
    \infty & \text{if } x = 0 \\
    0 & \text{if } x \bmod 2 = 1 \\
    1 + \texttt{wasted}(x \div 2) & \text{if } x \bmod 2 = 0 \\
  \end{cases}
\end{equation*}

\clearpage

\subsection{Computing Best \texttt{DIFF} Command, Energy and Residuals}
\label{shorten:compute_best_diff}
\ALGORITHM{a list of samples for a given channel and the channel's current block size}{a \texttt{DIFF} command, unsigned energy value and list of residuals}
\SetKwData{BLOCKSIZE}{block size}
\SetKwData{SAMPLE}{sample}
\SetKwData{DELTA}{delta}
\SetKwData{ENERGY}{energy}
\SetKwData{SUM}{sum}
\SetKwFunction{MIN}{min}
\For{$i \leftarrow -2$ \emph{\KwTo}\BLOCKSIZE}{
  $\text{\DELTA}_{1~i} \leftarrow \text{\SAMPLE}_i - \text{\SAMPLE}_{(i - 1)}$\;
}
\For{$i \leftarrow -1$ \emph{\KwTo}\BLOCKSIZE}{
  $\text{\DELTA}_{2~i} \leftarrow \text{\DELTA}_{1~i} - \text{\DELTA}_{1~(i - 1)}$\;
}
\For{$i \leftarrow 0$ \emph{\KwTo}\BLOCKSIZE}{
  $\text{\DELTA}_{3~i} \leftarrow \text{\DELTA}_{2~i} - \text{\DELTA}_{2~(i - 1)}$\;
}
\BlankLine
\For{$d \leftarrow 1$ \emph{\KwTo}4}{
  $\text{\SUM}_d \leftarrow \overset{\BLOCKSIZE - 1}{\underset{i = 0}{\sum}}|\text{\DELTA}_{d~i}|$\;
}
\BlankLine
$\ENERGY \leftarrow 0$\;
\uIf{$\text{\SUM}_1 < \MIN(\text{\SUM}_2~,~\text{\SUM}_3)$}{
  \While{$\BLOCKSIZE \times 2 ^ \text{\ENERGY} < \text{\SUM}_1$}{
    $\ENERGY \leftarrow \ENERGY + 1$\;
  }
  \Return $\left\lbrace\begin{tabular}{l}
  1 \\
  \ENERGY \\
  $\text{\DELTA}_{1~[0 \IDOTS \BLOCKSIZE]}$ \\
  \end{tabular}\right.$\;
}
\uElseIf{$\text{\SUM}_2 < \text{\SUM}_3$}{
  \While{$\BLOCKSIZE \times 2 ^ \text{\ENERGY} < \text{\SUM}_2$}{
    $\ENERGY \leftarrow \ENERGY + 1$\;
  }
  \Return $\left\lbrace\begin{tabular}{l}
  2 \\
  \ENERGY \\
  $\text{\DELTA}_{2~[0 \IDOTS \BLOCKSIZE]}$ \\
  \end{tabular}\right.$\;
}
\Else{
  \While{$\BLOCKSIZE \times 2 ^ \text{\ENERGY} < \text{\SUM}_3$}{
    $\ENERGY \leftarrow \ENERGY + 1$\;
  }
  \Return $\left\lbrace\begin{tabular}{l}
  3 \\
  \ENERGY \\
  $\text{\DELTA}_{3~[0 \IDOTS \BLOCKSIZE]}$ \\
  \end{tabular}\right.$\;
}
\EALGORITHM
\par
\noindent
Negative sample values are taken from the channel's previous samples,
or 0 if there are none.
Although negative delta values are needed for determining the next delta,
only the non-negative deltas are used for calculating the sums
and as returned residuals.

\clearpage

\subsubsection{Computing Best \texttt{DIFF} Command Example}
{\relsize{-1}
  \begin{tabular}{r|r|rrr}
    $i$ & $\textsf{sample}_i$ & $\textsf{delta}_{1~i}$ & $\textsf{delta}_{2~i}$ & $\textsf{delta}_{3~i}$ \\
    \hline
    \hline
    -3 & 0 & & & \\
    -2 & 0 & 0 & & \\
    -1 & 0 & 0 & 0 & \\
    \hline
    0 & 0 & 0 & 0 & 0 \\
    1 & 16 & 16 & 16 & 16 \\
    2 & 31 & 15 & -1 & -17 \\
    3 & 44 & 13 & -2 & -1 \\
    4 & 54 & 10 & -3 & -1 \\
    5 & 61 & 7 & -3 & 0 \\
    6 & 64 & 3 & -4 & -1 \\
    7 & 63 & -1 & -4 & 0 \\
    8 & 58 & -5 & -4 & 0 \\
    9 & 49 & -9 & -4 & 0 \\
    10 & 38 & -11 & -2 & 2 \\
    11 & 24 & -14 & -3 & -1 \\
    12 & 8 & -16 & -2 & 1 \\
    13 & -8 & -16 & 0 & 2 \\
    14 & -24 & -16 & 0 & 0 \\
    \hline
    \hline
    \multicolumn{2}{r}{$\textsf{sum}_d$} & 152 & 48 & 42 \\
  \end{tabular}
\vskip 1em
\par
\noindent
Since the $\textsf{sum}_3$ value of 42 is the smallest,
we'll use a \texttt{DIFF3} command.
The loop for calculating the energy value is:
\begin{align*}
\text{(block size) } 15 \times 2 ^ 0 &< 42 \text{ ($\textsf{sum}_3$)} \\
15 \times 2 ^ 1 &< 42 \\
15 \times 2 ^ 2 &> 42 \\
\end{align*}
Which means the best energy value to use is 1, the residuals are:
\newline
\texttt{[0, 16, -17, -1, -1, 0, -1, 0, 0, 0, 2, -1, 1, 2, 0]}
\newline
and the entire \texttt{DIFF3} command is encoded as:
}
\begin{figure}[h]
\includegraphics{shorten/figures/block1.pdf}
\end{figure}
