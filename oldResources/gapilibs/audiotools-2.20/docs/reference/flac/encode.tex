%This work is licensed under the
%Creative Commons Attribution-Share Alike 3.0 United States License.
%To view a copy of this license, visit
%http://creativecommons.org/licenses/by-sa/3.0/us/ or send a letter to
%Creative Commons,
%171 Second Street, Suite 300,
%San Francisco, California, 94105, USA.

\section{FLAC Encoding}

The basic process for encoding a FLAC file is as follows:
\par
\noindent
\ALGORITHM{PCM frames, various encoding parameters:
\newline
{\relsize{-1}
\begin{tabular}{rll}
parameter & possible values & typical values \\
\hline
block size & a positive number of PCM frames & 1152 or 4096 \\
maximum LPC order & integer between 0 and 32, inclusive & 0, 6, 8 or 12 \\
minimum partition order & integer between 0 and 16, inclusive & 0 \\
maximum partition order & integer between 0 and 16, inclusive & 3, 4, 5 or 6 \\
maximum Rice parameter & 14 if bits-per-sample $\leq 16$, otherwise 30 & \\
try mid-side & true or false & \\
try adaptive mid-side & true or false & \\
QLP precision & $\begin{cases}
7 & \text{ if } 0 < \text{block size} \leq 192 \\
8 & \text{ if } 192 < \text{block size} \leq 384 \\
9 & \text{ if } 384 < \text{block size} \leq 576 \\
10 & \text{ if } 576 < \text{block size} \leq 1152 \\
11 & \text{ if } 1152 < \text{block size} \leq 2304 \\
12 & \text{ if } 2304 < \text{block size} \leq 4608 \\
13 & \text{ if } \text{block size} > 4608 \\
\end{cases}$ & \\
exhaustive model search & true or false & \\
\end{tabular}
}
}{an encoded FLAC file}
\SetKwData{BLOCKSIZE}{block size}
$\texttt{"fLaC"} \rightarrow$ \WRITE 4 bytes\;
\hyperref[flac:write_placeholder_blocks]{write placeholder STREAMINFO metadata block}\;
write PADDING metadata block\;
initialize stream's MD5 sum\;
\While{PCM frames remain}{
  take \BLOCKSIZE PCM frames from the input\;
  \hyperref[flac:update_md5_w]{update the stream's MD5 sum with that PCM data}\;
  \hyperref[flac:encode_frame]{encode a FLAC frame from PCM frames using the given encoding parameters}\;
  update STREAMINFO's values from the FLAC frame\;
}
return to the start of the file and rewrite the STREAMINFO metadata block\;
\EALGORITHM
\begin{figure}[h]
\includegraphics{flac/figures/stream3.pdf}
\end{figure}
\par
\noindent
All of the fields in the FLAC stream are big-endian.

\clearpage

\subsection{Writing Placeholder Metadata Blocks}
\label{flac:write_placeholder_blocks}
\ALGORITHM{input stream's attributes, a default block size}{1 or more metadata blocks to the FLAC file stream}
\SetKwData{BLOCKSIZE}{block size}
\SetKwData{SAMPLERATE}{sample rate}
\SetKwData{CHANNELS}{channel count}
\SetKwData{BPS}{bits per sample}
$0 \rightarrow$ \WRITE 1 unsigned bit\tcc*[r]{is last block}
$0 \rightarrow$ \WRITE 7 unsigned bits\tcc*[r]{STREAMINFO type}
$34 \rightarrow$ \WRITE 24 unsigned bits\tcc*[r]{STREAMINFO size}
$\BLOCKSIZE \rightarrow$ \WRITE 16 unsigned bits\tcc*[r]{minimum block size}
$\BLOCKSIZE \rightarrow$ \WRITE 16 unsigned bits\tcc*[r]{maximum block size}
$0 \rightarrow$ \WRITE 24 unsigned bits\tcc*[r]{minimum frame size}
$0 \rightarrow$ \WRITE 24 unsigned bits\tcc*[r]{maximum frame size}
$\SAMPLERATE \rightarrow$ \WRITE 20 unsigned bits\;
$\CHANNELS - 1 \rightarrow$ \WRITE 3 unsigned bits\;
$\BPS - 1 \rightarrow$ \WRITE 5 unsigned bits\;
$0 \rightarrow$ \WRITE 36 unsigned bits\tcc*[r]{total PCM frames}
$0 \rightarrow$ \WRITE 16 bytes\tcc*[r]{stream's MD5 sum}
\BlankLine
\BlankLine
$1 \rightarrow$ \WRITE 1 unsigned bit\tcc*[r]{is last block}
$1 \rightarrow$ \WRITE 7 unsigned bits\tcc*[r]{PADDING type}
$4096 \rightarrow$ \WRITE 24 unsigned bits\tcc*[r]{PADDING size}
$0 \rightarrow$ \WRITE 4096 bytes\tcc*[r]{PADDING's data}
\EALGORITHM
\par
\noindent
PADDING can be some size other than 4096 bytes.
One simply wants to leave enough room for a VORBIS\_COMMENT block,
SEEKTABLE and so forth.
Other fields such as the minimum/maximum frame size
and the stream's final MD5 sum can't be known in advance;
we'll need to return to this block once encoding is finished
in order to populate them.
\begin{figure}[h]
\includegraphics{flac/figures/metadata.pdf}
\end{figure}


\clearpage

\subsection{Updating Stream MD5 Sum}
\label{flac:update_md5_w}
\ALGORITHM{the frame's signed PCM input samples\footnote{$channel_{c~i}$ indicates the $i$th sample in channel $c$}}{the stream's updated MD5 sum}
\SetKwData{BLOCKSIZE}{block size}
\SetKwData{CHANNELCOUNT}{channel count}
\SetKwData{CHANNEL}{channel}
\For{$i \leftarrow 0$ \emph{\KwTo}\BLOCKSIZE}{
  \For{$c \leftarrow 0$ \emph{\KwTo}\CHANNELCOUNT}{
    bytes $\leftarrow \text{\CHANNEL}_{c~i}$ as signed, little-endian bytes\;
    update stream's MD5 sum with bytes\;
  }
}
\Return stream's MD5 sum\;
\EALGORITHM
\par
\noindent
For example, given a 16 bits per sample stream with the signed sample values:
\begin{table}[h]
\begin{tabular}{r|rr}
$i$ & $\textsf{channel}_0$ & $\textsf{channel}_1$ \\
\hline
0 & 1 & -1 \\
1 & 2 & -2 \\
2 & 3 & -3 \\
\end{tabular}
\end{table}
\par
\noindent
are translated to the bytes:
\begin{table}[h]
\begin{tabular}{r|rr}
$i$ & $\textsf{channel}_0$ & $\textsf{channel}_1$ \\
\hline
0 & \texttt{01 00} & \texttt{FF FF} \\
1 & \texttt{02 00} & \texttt{FE FF} \\
2 & \texttt{03 00} & \texttt{FD FF} \\
\end{tabular}
\end{table}
\par
\noindent
and combined as:
\vskip .15in
\par
\noindent
\texttt{01 00 FF FF 02 00 FE FF 03 00 FD FF}
\vskip .15in
\par
\noindent
whose MD5 sum is:
\vskip .15in
\par
\noindent
\texttt{E7482f6462B27EE04EADC079291C79E9}
\vskip .25in
\par
This process is identical to the MD5 sum calculation performed
during FLAC decoding, but performed in the opposite order.

\clearpage

\subsection{Encoding a FLAC Frame}
\label{flac:encode_frame}
{\relsize{-1}
\ALGORITHM{up to ``block size'' number of PCM frames, encoding parameters}{a single FLAC frame}
\SetKw{AND}{and}
\SetKw{OR}{or}
\SetKwData{CHANNELCOUNT}{channel count}
\SetKwData{BPS}{bits per sample}
\SetKwData{CHANNEL}{channel}
\SetKwData{AVERAGE}{average}
\SetKwData{DIFFERENCE}{difference}
\SetKwData{LEFTS}{left subframe}
\SetKwData{RIGHTS}{right subframe}
\SetKwData{AVGS}{average subframe}
\SetKwData{DIFFS}{difference subframe}
\SetKwData{IBITS}{independent}
\SetKwData{LDBITS}{left/difference}
\SetKwData{DRBITS}{difference/right}
\SetKwData{ADBITS}{average/difference}
\SetKwData{SUBFRAME}{subframe}
\SetKwFunction{LEN}{len}
\SetKwFunction{MIN}{min}
\SetKwFunction{BUILDSUBFRAME}{build subframe}
\eIf{$\CHANNELCOUNT = 2$ \AND (try mid-side \OR try adaptive mid-side)}{
  \begin{tabular}{rcl}
    $(\AVERAGE~,~\DIFFERENCE)$ & $\leftarrow$ & \hyperref[flac:calc_midside]{calculate average-difference of $\text{\CHANNEL}_0$ and $\text{\CHANNEL}_1$} \\
    \LEFTS & $\leftarrow$ & \hyperref[flac:encode_subframe]{encode $\text{\CHANNEL}_0$ as subframe at $\BPS$} \\
    \RIGHTS & $\leftarrow$ & \hyperref[flac:encode_subframe]{encode $\text{\CHANNEL}_1$ as subframe at $\BPS$} \\
    \AVGS & $\leftarrow$ &  \hyperref[flac:encode_subframe]{encode $\text{\AVERAGE}$ as subframe at $\BPS$} \\
    \DIFFS & $\leftarrow$ & \hyperref[flac:encode_subframe]{encode $\text{\DIFFERENCE}$ as subframe at $(\BPS + 1)$} \\
    \IBITS & $\leftarrow$ & $\LEN(\LEFTS) + \LEN(\RIGHTS)$ \\
    \LDBITS & $\leftarrow$ & $\LEN(\LEFTS) + \LEN(\DIFFS)$ \\
    \DRBITS & $\leftarrow$ & $\LEN(\DIFFS) + \LEN(\RIGHTS)$ \\
    \ADBITS & $\leftarrow$ & $\LEN(\AVGS) + \LEN(\DIFFS)$ \\
  \end{tabular}\;
  \BlankLine
  \uIf{try mid-side}{
    \uIf{$\IBITS < \MIN(\LDBITS~,~\DRBITS~,~\ADBITS)$}{
      \hyperref[flac:write_frame_header]{write frame header with channel assignment \texttt{0x1}}\;
      write \LEFTS\;
      write \RIGHTS\;
    }
    \uElseIf{$\LDBITS < \MIN(\DRBITS~,~\ADBITS)$}{
      \hyperref[flac:write_frame_header]{write frame header with channel assignment \texttt{0x8}}\;
      write \LEFTS\;
      write \DIFFS\;
    }
    \uElseIf{$\DRBITS < \ADBITS$}{
      \hyperref[flac:write_frame_header]{write frame header with channel assignment \texttt{0x9}}\;
      write \DIFFS\;
      write \RIGHTS\;
    }
    \Else{
      \hyperref[flac:write_frame_header]{write frame header with channel assignment \texttt{0xA}}\;
      write \AVGS\;
      write \DIFFS\;
    }
  }\uElseIf{$\IBITS < \ADBITS$}{
    \hyperref[flac:write_frame_header]{write frame header with channel assignment \texttt{0x1}}\;
    write \LEFTS\;
    write \RIGHTS\;
  }
  \Else{
    \hyperref[flac:write_frame_header]{write frame header with channel assignment \texttt{0xA}}\;
    write \AVGS\;
    write \DIFFS\;
  }
}(\tcc*[f]{store subframes independently}){
  \hyperref[flac:write_frame_header]{write frame header with channel assignment $\CHANNELCOUNT - 1$}\;
  \For{$c \leftarrow 0$ \emph{\KwTo}\CHANNELCOUNT}{
    $\text{\SUBFRAME}_c \leftarrow $ \hyperref[flac:encode_subframe]{encode $\text{\CHANNEL}_c$ as subframe at $\BPS$}\;
    write $\text{\SUBFRAME}_c$\;
  }
}
byte align the stream\;
\hyperref[flac:calculate_crc16]{write frame's CRC-16 checksum}\;
\EALGORITHM
}

\clearpage

\subsubsection{Calculating Average-Difference}
\label{flac:calc_midside}
\ALGORITHM{block size, 2 channels of PCM data}{2 channels stored as average / difference}
\SetKwData{BLOCKSIZE}{block size}
\SetKwData{CHANNEL}{channel}
\SetKwData{AVERAGE}{average}
\SetKwData{DIFFERENCE}{difference}
\For{$i \leftarrow 0$ \emph{\KwTo}\BLOCKSIZE}{
  $\text{\AVERAGE}_i \leftarrow \lfloor (\text{\CHANNEL}_{0~i} + \text{\CHANNEL}_{1~i}) \div 2\rfloor$\;
  $\text{\DIFFERENCE}_i \leftarrow \text{\CHANNEL}_{0~i} - \text{\CHANNEL}_{1~i}$\;
}
\Return $\left\lbrace\begin{tabular}{l}
\AVERAGE \\
\DIFFERENCE \\
\end{tabular}\right.$\;
\EALGORITHM
\begin{align*}
\intertext{For example, given the input samples:}
\textsf{channel}_{0~0} &\leftarrow 10 \\
\textsf{channel}_{1~0} &\leftarrow 15
\intertext{Our average and difference samples are:}
\textsf{average}_0 &\leftarrow \left\lfloor\frac{10 + 15}{2}\right\rfloor = 12 \\
\textsf{difference}_0 &\leftarrow 10 - 15 = -5
\intertext{Note that the \textsf{difference} channel is identical
for left-difference, difference-right and average-difference
channel assignments.
For example, when recombined from left-difference\footnotemark:}
\textsf{sample}_0 &\leftarrow 10 \\
\textsf{sample}_1 &\leftarrow 10 - (-5) = 15
\intertext{difference-right:}
\textsf{sample}_0 &\leftarrow -5 + 15 = 10 \\
\textsf{sample}_1 &\leftarrow 15
\intertext{and average-difference:}
\textsf{sample}_0 &\leftarrow \lfloor(((12 \times 2) + (-5 \bmod 2)) + -5) \div 2\rfloor  = \lfloor((24 + 1 - 5) \div 2\rfloor = 10 \\
\textsf{sample}_1 &\leftarrow \lfloor(((12 \times 2) + (-5 \bmod 2)) - -5) \div 2\rfloor =  \lfloor((24 + 1 + 5) \div 2\rfloor = 15
\end{align*}
\footnotetext{See the recombining subframes algorithms on page
\pageref{flac:recombine_subframes}.}

\clearpage

\subsubsection{Writing Frame Header}
\label{flac:write_frame_header}
{\relsize{-1}
\ALGORITHM{the frame's channel assignment, the input stream's parameters}{a FLAC frame header}
\SetKwData{BLOCKSIZE}{block size}
\SetKwData{SAMPLERATE}{sample rate}
\SetKwData{FRAMENUMBER}{frame number}
\SetKwData{EBLOCKSIZE}{encoded block size}
\SetKwData{ESAMPLERATE}{encoded sample rate}
\SetKwData{EBPS}{encoded bits per sample}
\SetKwData{ASSIGNMENT}{channel assignment}
\SetKwData{CRC}{CRC-8}
\SetKw{OR}{or}
$\texttt{0x3FFE} \rightarrow$ \WRITE 14 unsigned bits\tcc*[r]{sync code}
$0 \rightarrow$ \WRITE 1 unsigned bit\;
$0 \rightarrow$ \WRITE 1 unsigned bit\tcc*[r]{blocking strategy}
$\EBLOCKSIZE \rightarrow$ \WRITE 4 unsigned bits\;
$\ESAMPLERATE \rightarrow$ \WRITE 4 unsigned bits\;
$\ASSIGNMENT \rightarrow$ \WRITE 4 unsigned bits\;
$\EBPS \rightarrow$ \WRITE 3 unsigned bits\;
$0 \rightarrow$ \WRITE 1 unsigned bit\;
$\FRAMENUMBER \rightarrow$ \WRITE \hyperref[flac:write_utf8]{as UTF-8 encoded value}\;
\uIf{$\EBLOCKSIZE = 6$}{
  $(\BLOCKSIZE - 1) \rightarrow$ \WRITE 8 unsigned bits\;
}
\ElseIf{$\EBLOCKSIZE = 7$}{
  $\BLOCKSIZE - 1 \rightarrow$ \WRITE 16 unsigned bits\;
}
\uIf{$\ESAMPLERATE = 12$}{
  $\SAMPLERATE \div 1000 \rightarrow$ \WRITE 8 unsigned bits\;
}
\uElseIf{$\ESAMPLERATE = 13$}{
  $\SAMPLERATE \rightarrow$ \WRITE 16 unsigned bits\;
}
\ElseIf{$\ESAMPLERATE = 14$}{
  $\SAMPLERATE \div 10 \rightarrow$ \WRITE 16 unsigned bits\;
}
$\CRC \leftarrow$ \hyperref[flac:calculate_crc8]{calculate frame header's CRC-8}\;
$\CRC \rightarrow$ \WRITE 8 unsigned bits\;
\EALGORITHM
}

\subsubsection{Encoding Block Size}
{\relsize{-1}
\ALGORITHM{block size in samples}{encoded block size as 4 bit value}
\SetKwData{BLOCKSIZE}{block size}
\Switch{\BLOCKSIZE}{
\lCase{192}{\Return 1}\;
\lCase{256}{\Return 8}\;
\lCase{512}{\Return 9}\;
\lCase{576}{\Return 2}\;
\lCase{1024}{\Return 10}\;
\lCase{1152}{\Return 3}\;
\lCase{2048}{\Return 11}\;
\lCase{2304}{\Return 4}\;
\lCase{4096}{\Return 12}\;
\lCase{4608}{\Return 5}\;
\lCase{8192}{\Return 13}\;
\lCase{16384}{\Return 14}\;
\lCase{32768}{\Return 15}\;
\Other{
  \lIf{$\BLOCKSIZE \leq 256$}{\Return 6}\;
  \lElseIf{$\BLOCKSIZE \leq 65536$}{\Return 7}\;
  \lElse{\Return 0}
}
}
\EALGORITHM
}

\clearpage

\subsubsection{Encoding Sample Rate}
{\relsize{-1}
\ALGORITHM{sample rate in Hz}{encoded sample rate as 4 bit value}
\SetKw{AND}{and}
\SetKwData{SAMPLERATE}{sample rate}
\Switch{\SAMPLERATE}{
\lCase{8000}{\Return 4}\;
\lCase{16000}{\Return 5}\;
\lCase{22050}{\Return 6}\;
\lCase{24000}{\Return 7}\;
\lCase{32000}{\Return 8}\;
\lCase{44100}{\Return 9}\;
\lCase{48000}{\Return 10}\;
\lCase{88200}{\Return 1}\;
\lCase{96000}{\Return 11}\;
\lCase{176400}{\Return 2}\;
\lCase{192000}{\Return 3}\;
\Other{
  \lIf{$(\SAMPLERATE \bmod 1000 = 0)$ \AND $(\SAMPLERATE \leq 255000)$}{\Return 12}\;
  \lElseIf{$(\SAMPLERATE \bmod 10 = 0)$ \AND $(\SAMPLERATE \leq 655350)$}{\Return 14}\;
  \lElseIf{$\SAMPLERATE \leq 65535$}{\Return 13}\;
  \lElse{\Return 0}
}
}
\EALGORITHM
}
\subsubsection{Encoding Bits Per Sample}
{\relsize{-1}
\ALGORITHM{bits per sample}{encoded bits per sample as 3 bit value}
\SetKwData{BPS}{bits per sample}
\Switch{\BPS}{
\lCase{8}{\Return 1}\;
\lCase{12}{\Return 2}\;
\lCase{16}{\Return 4}\;
\lCase{20}{\Return 5}\;
\lCase{24}{\Return 6}\;
\lOther{\Return 0}\;
}
\EALGORITHM
}
\begin{figure}[h]
\includegraphics{flac/figures/frames.pdf}
\end{figure}

\clearpage

\subsubsection{Encoding UTF-8 Frame Number}
\label{flac:write_utf8}
{\relsize{-1}
\ALGORITHM{value as unsigned integer}{1 or more UTF-8 bytes}
\SetKwData{VALUE}{value}
\SetKwData{TOTALBYTES}{total bytes}
\SetKwData{SHIFT}{shift}
\eIf{$\VALUE \leq 127$}{
  $\VALUE \rightarrow$ \WRITE 8 unsigned bits\;
}{
  \uIf{$\VALUE \leq 2047$}{
    $\TOTALBYTES \leftarrow 2$\;
  }
  \uElseIf{$\VALUE \leq 65535$}{
    $\TOTALBYTES \leftarrow 3$\;
  }
  \uElseIf{$\VALUE \leq 2097151$}{
    $\TOTALBYTES \leftarrow 4$\;
  }
  \uElseIf{$\VALUE \leq 67108863$}{
    $\TOTALBYTES \leftarrow 5$\;
  }
  \ElseIf{$\VALUE \leq 2147483647$}{
    $\TOTALBYTES \leftarrow 6$\;
  }
  $\SHIFT \leftarrow (\TOTALBYTES - 1) \times 6$\;
  $\TOTALBYTES \rightarrow$ \WUNARY with stop bit 0\;
  $\lfloor \text{\VALUE} \div 2 ^ \text{\SHIFT} \rfloor \rightarrow$ \WRITE $(7 - \TOTALBYTES)$ unsigned bits\tcc*[r]{initial value}
  $\SHIFT \leftarrow \SHIFT - 6$\;
  \While{$\SHIFT \geq 0$}{
    $2 \rightarrow$ \WRITE 2 unsigned bits\tcc*[r]{continuation header}
    $\lfloor \VALUE \div 2 ^ \text{\SHIFT} \rfloor \bmod 64 \rightarrow$ \WRITE 6 unsigned bits\tcc*[r]{continuation bits}
    $\SHIFT \leftarrow \SHIFT - 6$\;
  }
}
\EALGORITHM
}
\par
\noindent
For example, encoding the frame number 4228 in UTF-8:
\par
\noindent
\begin{wrapfigure}[10]{r}{2.375in}
\includegraphics{flac/figures/utf8.pdf}
\end{wrapfigure}
\begin{align*}
\textsf{total bytes} &\leftarrow 3 \\
\textsf{shift} &\leftarrow 12 \\
& 3 \rightarrow \textbf{write unary} \text{ with stop bit 1} \\
& 1 \rightarrow \textbf{write} \text{ in 4 unsigned bits} \\
\textsf{shift} &\leftarrow 12 - 6 = 6 \\
& 2 \rightarrow \textbf{write} \text{ in 2 unsigned bits} \\
& 2 \rightarrow \textbf{write} \text{ in 6 unsigned bits} \\
\textsf{shift} &\leftarrow 6 - 6 = 0 \\
& 2 \rightarrow \textbf{write} \text{ in 2 unsigned bits} \\
& 4 \rightarrow \textbf{write} \text{ in 6 unsigned bits}
\end{align*}

\clearpage

\subsubsection{Calculating CRC-8}
\label{flac:calculate_crc8}
Given a header byte and previous CRC-8 checksum,
or 0 as an initial value:
\begin{equation*}
\text{checksum}_i = \text{CRC8}(byte\xor\text{checksum}_{i - 1})
\end{equation*}
\begin{table}[h]
{\relsize{-3}\ttfamily
\begin{tabular}{|r||r|r|r|r|r|r|r|r|r|r|r|r|r|r|r|r|r|}
\hline
 & 0x?0 & 0x?1 & 0x?2 & 0x?3 & 0x?4 & 0x?5 & 0x?6 & 0x?7 & 0x?8 & 0x?9 & 0x?A & 0x?B & 0x?C & 0x?D & 0x?E & 0x?F \\
\hline
0x0? & 0x00 & 0x07 & 0x0E & 0x09 & 0x1C & 0x1B & 0x12 & 0x15 & 0x38 & 0x3F & 0x36 & 0x31 & 0x24 & 0x23 & 0x2A & 0x2D \\
0x1? & 0x70 & 0x77 & 0x7E & 0x79 & 0x6C & 0x6B & 0x62 & 0x65 & 0x48 & 0x4F & 0x46 & 0x41 & 0x54 & 0x53 & 0x5A & 0x5D \\
0x2? & 0xE0 & 0xE7 & 0xEE & 0xE9 & 0xFC & 0xFB & 0xF2 & 0xF5 & 0xD8 & 0xDF & 0xD6 & 0xD1 & 0xC4 & 0xC3 & 0xCA & 0xCD \\
0x3? & 0x90 & 0x97 & 0x9E & 0x99 & 0x8C & 0x8B & 0x82 & 0x85 & 0xA8 & 0xAF & 0xA6 & 0xA1 & 0xB4 & 0xB3 & 0xBA & 0xBD \\
0x4? & 0xC7 & 0xC0 & 0xC9 & 0xCE & 0xDB & 0xDC & 0xD5 & 0xD2 & 0xFF & 0xF8 & 0xF1 & 0xF6 & 0xE3 & 0xE4 & 0xED & 0xEA \\
0x5? & 0xB7 & 0xB0 & 0xB9 & 0xBE & 0xAB & 0xAC & 0xA5 & 0xA2 & 0x8F & 0x88 & 0x81 & 0x86 & 0x93 & 0x94 & 0x9D & 0x9A \\
0x6? & 0x27 & 0x20 & 0x29 & 0x2E & 0x3B & 0x3C & 0x35 & 0x32 & 0x1F & 0x18 & 0x11 & 0x16 & 0x03 & 0x04 & 0x0D & 0x0A \\
0x7? & 0x57 & 0x50 & 0x59 & 0x5E & 0x4B & 0x4C & 0x45 & 0x42 & 0x6F & 0x68 & 0x61 & 0x66 & 0x73 & 0x74 & 0x7D & 0x7A \\
0x8? & 0x89 & 0x8E & 0x87 & 0x80 & 0x95 & 0x92 & 0x9B & 0x9C & 0xB1 & 0xB6 & 0xBF & 0xB8 & 0xAD & 0xAA & 0xA3 & 0xA4 \\
0x9? & 0xF9 & 0xFE & 0xF7 & 0xF0 & 0xE5 & 0xE2 & 0xEB & 0xEC & 0xC1 & 0xC6 & 0xCF & 0xC8 & 0xDD & 0xDA & 0xD3 & 0xD4 \\
0xA? & 0x69 & 0x6E & 0x67 & 0x60 & 0x75 & 0x72 & 0x7B & 0x7C & 0x51 & 0x56 & 0x5F & 0x58 & 0x4D & 0x4A & 0x43 & 0x44 \\
0xB? & 0x19 & 0x1E & 0x17 & 0x10 & 0x05 & 0x02 & 0x0B & 0x0C & 0x21 & 0x26 & 0x2F & 0x28 & 0x3D & 0x3A & 0x33 & 0x34 \\
0xC? & 0x4E & 0x49 & 0x40 & 0x47 & 0x52 & 0x55 & 0x5C & 0x5B & 0x76 & 0x71 & 0x78 & 0x7F & 0x6A & 0x6D & 0x64 & 0x63 \\
0xD? & 0x3E & 0x39 & 0x30 & 0x37 & 0x22 & 0x25 & 0x2C & 0x2B & 0x06 & 0x01 & 0x08 & 0x0F & 0x1A & 0x1D & 0x14 & 0x13 \\
0xE? & 0xAE & 0xA9 & 0xA0 & 0xA7 & 0xB2 & 0xB5 & 0xBC & 0xBB & 0x96 & 0x91 & 0x98 & 0x9F & 0x8A & 0x8D & 0x84 & 0x83 \\
0xF? & 0xDE & 0xD9 & 0xD0 & 0xD7 & 0xC2 & 0xC5 & 0xCC & 0xCB & 0xE6 & 0xE1 & 0xE8 & 0xEF & 0xFA & 0xFD & 0xF4 & 0xF3 \\
\hline
\end{tabular}
}
\end{table}

\subsubsection{Frame Header Encoding Example}
Given a frame header with the following attributes:
\begin{table}[h]
\begin{tabular}{rl}
block size : & 4096 PCM frames \\
sample rate : & 44100 Hz \\
channel assignment : & 1 (2 channels stored independently) \\
bits per sample : & 16 \\
frame number : & 0
\end{tabular}
\end{table}
\par
\noindent
we generate the following frame header bytes:
\begin{figure}[h]
\includegraphics{flac/figures/header-example.pdf}
\end{figure}
\par
\noindent
Note how the CRC-8 is calculated from the preceding 5 header bytes:
\begin{align*}
\text{checksum}_0 = \text{CRC8}(\texttt{FF}\xor\texttt{00}) = \texttt{F3} & &
\text{checksum}_3 = \text{CRC8}(\texttt{18}\xor\texttt{E6}) = \texttt{F4} \\
\text{checksum}_1 = \text{CRC8}(\texttt{F8}\xor\texttt{F3}) = \texttt{31} & &
\text{checksum}_4 = \text{CRC8}(\texttt{00}\xor\texttt{F4}) = \texttt{C2} \\
\text{checksum}_2 = \text{CRC8}(\texttt{C9}\xor\texttt{31}) = \texttt{E6} \\
\end{align*}

\clearpage

\subsection{Encoding a FLAC Subframe}
\label{flac:encode_subframe}
{\relsize{-1}
\ALGORITHM{block size, signed subframe samples, subframe's bits per sample}{a FLAC subframe}
\SetKwData{BLOCKSIZE}{block size}
\SetKwData{SAMPLE}{sample}
\SetKwData{WASTEDBPS}{wasted BPS}
\SetKwData{BPS}{subframe's BPS}
\SetKwData{CONSTANT}{CONSTANT subframe}
\SetKwData{FIXED}{FIXED subframe}
\SetKwData{VERBATIM}{VERBATIM subframe}
\SetKwData{ORDER}{LPC order}
\SetKwData{QLPPRECISION}{QLP precision}
\SetKwData{QLPSHIFT}{QLP shift needed}
\SetKwData{QLPCOEFFS}{QLP coefficients}
\SetKwData{LPC}{LPC subframe}
\SetKwFunction{LEN}{len}
\SetKwFunction{MIN}{min}
\eIf{all samples are the same}{
  \Return \hyperref[flac:encode_constant_subframe]{\CONSTANT from}
  $\left\lbrace\begin{tabular}{l}
  $\text{\SAMPLE}_0$ \\
  $\BPS$ \\
  \end{tabular}\right.$\;
}{
  $\WASTEDBPS \leftarrow$ \hyperref[flac:calculate_wasted_bps]{calculate wasted bits per sample for $\text{\SAMPLE}$}\;
  \If{$\WASTEDBPS > 0$}{
    \For{$i \leftarrow 0$ \emph{\KwTo}\BLOCKSIZE}{
      $\text{\SAMPLE}_i \leftarrow \text{\SAMPLE}_i \div 2 ^ \text{\WASTEDBPS}$\;
    }
    $\BPS \leftarrow \BPS - \WASTEDBPS$\;
  }
  $\VERBATIM \leftarrow$ \hyperref[flac:encode_verbatim_subframe]{build VERBATIM subframe from}
  $\left\lbrace\begin{tabular}{l}
  \SAMPLE \\
  \BLOCKSIZE \\
  \BPS \\
  \WASTEDBPS \\
  \end{tabular}\right.$\;
  \BlankLine
  $\FIXED \leftarrow$ \hyperref[flac:encode_fixed_subframe]{build FIXED subframe from}
  $\left\lbrace\begin{tabular}{l}
  \SAMPLE \\
  \BLOCKSIZE \\
  \BPS \\
  \WASTEDBPS \\
  \end{tabular}\right.$\;
  \BlankLine
  \eIf(\tcc*[f]{from encoding parameters}){maximum LPC order $ > 0$}{
    $\left.\begin{tabular}{r}
      $\ORDER$ \\
      $\QLPPRECISION$ \\
      $\QLPSHIFT$ \\
      $\QLPCOEFFS$ \\
      \end{tabular}\right\rbrace \leftarrow$
    \hyperref[flac:compute_lpc_params]{compute LPC parameters from \SAMPLE and \BLOCKSIZE}\;
    $\text{\LPC} \leftarrow$ \hyperref[flac:encode_lpc_subframe]{build LPC subframe from}
    $\left\lbrace\begin{tabular}{l}
    \SAMPLE \\
    \BLOCKSIZE \\
    \BPS \\
    \WASTEDBPS \\
    \ORDER \\
    \QLPPRECISION \\
    \QLPSHIFT \\
    \QLPCOEFFS \\
    \end{tabular}\right.$\;
    \BlankLine
    \uIf{$\LEN(\VERBATIM) \leq \MIN(\LEN(\LPC)~,~\LEN(\FIXED))$}{
      \Return \VERBATIM\;
    }
    \uElseIf{$\LEN(\FIXED) \leq \LEN(\LPC)$}{
      \Return \FIXED\;
    }
    \Else{
      \Return \LPC\;
    }
  }{
    \eIf{$\LEN(\VERBATIM) \leq \LEN(\FIXED)$}{
      \Return \VERBATIM\;
    }{
      \Return \FIXED\;
    }
  }
}
\EALGORITHM
}

\clearpage

\subsubsection{Calculating Wasted Bits Per Sample}
\label{flac:calculate_wasted_bps}
\ALGORITHM{a list of signed PCM samples}{an unsigned integer}
\SetKwData{WASTEDBPS}{wasted bps}
\SetKwData{SAMPLE}{sample}
\SetKwData{BLOCKSIZE}{block size}
\SetKwFunction{MIN}{min}
\SetKwFunction{WASTED}{wasted}
$\text{\WASTEDBPS} \leftarrow \infty$\tcc*[r]{maximum unsigned integer}
\For{$i \leftarrow 0$ \emph{\KwTo}\BLOCKSIZE}{
  $\text{\WASTEDBPS} \leftarrow \MIN(\WASTED(\text{\SAMPLE}_i)~,~\text{\WASTEDBPS})$\;
  \If{$\text{\WASTEDBPS} = 0$}{
    \Return 0\;
  }
}
\eIf(\tcc*[f]{all samples are 0}){$\WASTEDBPS = \infty$}{
  \Return 0\;
}{
  \Return \WASTEDBPS\;
}
\EALGORITHM
where the \texttt{wasted} function is defined as:
\begin{equation*}
  \texttt{wasted}(x) =
  \begin{cases}
    \infty & \text{if } x = 0 \\
    0 & \text{if } x \bmod 2 = 1 \\
    1 + \texttt{wasted}(x \div 2) & \text{if } x \bmod 2 = 0 \\
  \end{cases}
\end{equation*}

\clearpage

\subsection{Encoding a CONSTANT Subframe}
\label{flac:encode_constant_subframe}
\ALGORITHM{signed subframe sample, subframe's bits per sample}{a CONSTANT subframe}
\SetKwData{SAMPLE}{sample}
\SetKwData{BPS}{subframe's BPS}
$0 \rightarrow$ \WRITE 1 unsigned bit\tcc*[r]{pad}
$0 \rightarrow$ \WRITE 6 unsigned bits\tcc*[r]{subframe type}
$0 \rightarrow$ \WRITE 1 unsigned bit\tcc*[r]{no wasted BPS}
$\text{\SAMPLE} \rightarrow$ \WRITE $(\BPS)$ signed bits\;
\Return a CONSTANT subframe\;
\EALGORITHM
\begin{figure}[h]
  \includegraphics{flac/figures/constant.pdf}
\end{figure}

\clearpage

\subsection{Encoding a VERBATIM Subframe}
\label{flac:encode_verbatim_subframe}
\ALGORITHM{signed subframe samples, block size, subframe's bits per sample, wasted BPS}{a VERBATIM subframe}
\SetKwData{BLOCKSIZE}{block size}
\SetKwData{SAMPLE}{sample}
\SetKwData{BPS}{subframe's BPS}
\SetKwData{WASTEDBPS}{wasted BPS}
$0 \rightarrow$ \WRITE 1 unsigned bit\tcc*[r]{pad}
$1 \rightarrow$ \WRITE 6 unsigned bits\tcc*[r]{subframe type}
\eIf{$\WASTEDBPS > 0$}{
  $1 \rightarrow$ \WRITE 1 unsigned bit\;
  $(\text{\WASTEDBPS} - 1) \rightarrow$ \WUNARY with stop bit 1\;
}{
  $0 \rightarrow$ \WRITE 1 unsigned bit\;
}
\For{$i \leftarrow 0$ \emph{\KwTo}\BLOCKSIZE}{
  $\text{\SAMPLE}_i \rightarrow$ \WRITE $(\BPS)$ signed bits\;
}
\Return a VERBATIM subframe\;
\EALGORITHM
\begin{figure}[h]
  \includegraphics{flac/figures/verbatim.pdf}
\end{figure}

\clearpage

%This work is licensed under the
%Creative Commons Attribution-Share Alike 3.0 United States License.
%To view a copy of this license, visit
%http://creativecommons.org/licenses/by-sa/3.0/us/ or send a letter to
%Creative Commons,
%171 Second Street, Suite 300,
%San Francisco, California, 94105, USA.

\subsection{Encoding a FIXED Subframe}
\label{flac:encode_fixed_subframe}
{\relsize{-1}
\ALGORITHM{signed subframe samples, block size, subframe's bits per sample, wasted BPS}{a FIXED subframe}
\SetKwData{BPS}{subframe's BPS}
\SetKwData{BLOCKSIZE}{block size}
\SetKwData{RESIDUAL}{residual}
\SetKwData{SAMPLE}{sample}
\SetKwData{ERROR}{total error}
\SetKwData{ORDER}{order}
\SetKwData{WASTEDBPS}{wasted BPS}
\tcc{first decide which FIXED subframe order to use}
\For(\tcc*[f]{order 0}){$i \leftarrow 0$ \emph{\KwTo}\BLOCKSIZE}{
  $\text{\RESIDUAL}_{0~i} \leftarrow \text{\SAMPLE}_i$\;
}
$\text{\ERROR}_0 \leftarrow \overset{\BLOCKSIZE - 1}{\underset{i = 4}{\sum}}|\text{\RESIDUAL}_{0~i}|$\;
\BlankLine
\eIf{$\BLOCKSIZE > 4$}{
\For(\tcc*[f]{order 1-4}){$\ORDER \leftarrow 1$ \emph{\KwTo}5}{
  \For{$i \leftarrow 0$ \emph{\KwTo}$\BLOCKSIZE - \ORDER$}{
  $\text{\RESIDUAL}_{\ORDER~i} \leftarrow \text{\RESIDUAL}_{(\ORDER - 1)~(i + 1)} - \text{\RESIDUAL}_{(\ORDER - 1)~i}$\;
  }
  $\text{\ERROR}_{\ORDER} \leftarrow \overset{\BLOCKSIZE - \ORDER - 1}{\underset{i = 4 - \ORDER}{\sum}}|\text{\RESIDUAL}_{\ORDER~i}|$\;
}
\BlankLine
choose subframe \ORDER such that $\text{\ERROR}_{\ORDER}$ is smallest\;
}{
use subframe \ORDER 0\;
}
\BlankLine
\tcc{then return a FIXED subframe with best order}
$0 \rightarrow$ \WRITE 1 unsigned bit\tcc*[r]{pad}
$1 \rightarrow$ \WRITE 3 unsigned bits\tcc*[r]{subframe type}
$\text{\ORDER} \rightarrow$ \WRITE 3 unsigned bits\;
\eIf{$\WASTEDBPS > 0$}{
  $1 \rightarrow$ \WRITE 1 unsigned bit\;
  $(\text{\WASTEDBPS} - 1) \rightarrow$ \WUNARY with stop bit 1\;
}{
  $0 \rightarrow$ \WRITE 1 unsigned bit\;
}
\For(\tcc*[f]{warm-up samples}){$i \leftarrow 0$ \emph{\KwTo}\ORDER}{
  $\text{\SAMPLE}_i \rightarrow$ \WRITE $(\BPS)$ signed bits\;
}
\hyperref[flac:write_residual_block]{write encoded residual block based on $\text{\RESIDUAL}_{\ORDER}$, \BLOCKSIZE and \ORDER}\;
\BlankLine
\Return a FIXED subframe\;
\EALGORITHM
}
\begin{figure}[h]
  \includegraphics{flac/figures/fixed.pdf}
\end{figure}

\clearpage

\subsubsection{FIXED Subframe Calculation Example}

Given the subframe samples: \texttt{18, 20, 26, 24, 24, 23, 21, 24, 23, 20}:
\begin{table}[h]
\begin{tabular}{r|r|r|r|r|r}
& \textsf{order} = 0 & \textsf{order} = 1 & \textsf{order} = 2 & \textsf{order} = 3 & \textsf{order} = 4 \\
\hline
$\textsf{residual}_{\textsf{order}~0}$ & \texttt{\color{gray}18} & \texttt{\color{gray}2} & \texttt{\color{gray}4} & \texttt{\color{gray}-12} & \texttt{22} \\
$\textsf{residual}_{\textsf{order}~1}$ & \texttt{\color{gray}20} & \texttt{\color{gray}6} & \texttt{\color{gray}-8} & \texttt{10} & \texttt{-13} \\
$\textsf{residual}_{\textsf{order}~2}$ & \texttt{\color{gray}26} & \texttt{\color{gray}-2} & \texttt{2} & \texttt{-3} & \texttt{3} \\
$\textsf{residual}_{\textsf{order}~3}$ & \texttt{\color{gray}24} & \texttt{0} & \texttt{-1} & \texttt{0} & \texttt{6} \\
$\textsf{residual}_{\textsf{order}~4}$ & \texttt{24} & \texttt{-1} & \texttt{-1} & \texttt{6} & \texttt{-15} \\
$\textsf{residual}_{\textsf{order}~5}$ & \texttt{23} & \texttt{-2} & \texttt{5} & \texttt{-9} & \texttt{11} \\
$\textsf{residual}_{\textsf{order}~6}$ & \texttt{21} & \texttt{3} & \texttt{-4} & \texttt{2} \\
$\textsf{residual}_{\textsf{order}~7}$ & \texttt{24} & \texttt{-1} & \texttt{-2} \\
$\textsf{residual}_{\textsf{order}~8}$ & \texttt{23} & \texttt{-3} \\
$\textsf{residual}_{\textsf{order}~9}$ & \texttt{20} \\
\hline
$\textsf{total error}_{\textsf{order}}$ & \texttt{135} & \texttt{10} & \texttt{15} & \texttt{30} & \texttt{70} \\
\end{tabular}
\end{table}
\par
\noindent
Note how the total number of residuals equals the
total number of samples minus the subframe's order,
to account for the warm-up samples.
Also note that if you remove the first $4 - \textsf{order}$ residuals
and sum the absolute value of the remaining residuals,
the result is the \VAR{total error} value
used when calculating the best FIXED subframe order.
\par
Since $\textsf{total error}_1$'s value of 10 is the smallest,
the best order for this FIXED subframe is 1.

\begin{figure}[h]
  \includegraphics{flac/figures/fixed-enc-example.pdf}
\end{figure}


\clearpage

%This work is licensed under the
%Creative Commons Attribution-Share Alike 3.0 United States License.
%To view a copy of this license, visit
%http://creativecommons.org/licenses/by-sa/3.0/us/ or send a letter to
%Creative Commons,
%171 Second Street, Suite 300,
%San Francisco, California, 94105, USA.

\subsection{Residual Encoding}
\label{flac:write_residual_block}
{\relsize{-1}
\ALGORITHM{a set of signed residual values, the subframe's block size and predictor order, minimum and maximum partition order from encoding parameters}{an encoded block of residuals}
\SetKwData{MINPORDER}{minimum partition order}
\SetKwData{MAXPORDER}{maximum partition order}
\SetKwData{ORDER}{predictor order}
\SetKwData{BLOCKSIZE}{block size}
\SetKwData{PAORDER}{partition order}
\SetKwData{PSIZE}{partitions size}
\SetKwData{PSUM}{partition sum}
\SetKwData{RICE}{Rice}
\SetKwData{CODING}{coding method}
\SetKwData{UNSIGNED}{unsigned}
\SetKwData{PARTITION}{partition}
\SetKwData{PLEN}{partition length}
\SetKwData{MSB}{MSB}
\SetKwData{LSB}{LSB}
\SetKwFunction{SUM}{sum}
\SetKwFunction{MAX}{max}
\SetKw{BREAK}{break}
\tcc{generate set of partitions for each partition order}
\For{$o \leftarrow \text{\MINPORDER}$ \emph{\KwTo}(\MAXPORDER + 1)}{
  \eIf{$(\BLOCKSIZE \bmod 2^{o}) = 0$}{
    $\left.\begin{tabular}{r}
      $\text{\RICE}_o$ \\
      $\text{\PARTITION}_o$ \\
      $\text{\PSIZE}_o$ \\
    \end{tabular}\right\rbrace \leftarrow$ \hyperref[flac:write_residual_partition]{encode residual partitions from}
    $\left\lbrace\begin{tabular}{l}
    partition order $o$ \\
    \textsf{predictor order} \\
    \textsf{residual values} \\
    \BLOCKSIZE \\
    \end{tabular}\right.$
  }{
    \BREAK\;
  }
}
\BlankLine
choose partition order $o$ such that $\PSIZE_{o}$ is smallest\;
\BlankLine
\eIf{$\MAX(\text{\RICE}_{o}) > 14$}{
  $\CODING \leftarrow 1$\;
}{
  $\CODING \leftarrow 0$\;
}
\BlankLine
\tcc{write 1 or more residual partitions to residual block}
$\CODING \rightarrow$ \WRITE 2 unsigned bits\;
$o \rightarrow$ \WRITE 4 unsigned bits\;
\For{$p \leftarrow 0$ \emph{\KwTo}$2 ^ {o}$} {
  \eIf{$\CODING = 0$}{
    $\text{\RICE}_{o~p} \rightarrow$ \WRITE 4 unsigned bits\;
  }{
    $\text{\RICE}_{o~p} \rightarrow$ \WRITE 5 unsigned bits\;
  }
  \BlankLine
  \eIf{$p = 0$}{
    $\text{\PLEN}_{o~0} \leftarrow \BLOCKSIZE \div 2 ^ {o} - \ORDER$\;
  }{
    $\text{\PLEN}_{o~p} \leftarrow \BLOCKSIZE \div 2 ^ {o}$\;
  }
  \BlankLine
  \For(\tcc*[f]{write residual partition}){$i \leftarrow 0$ \emph{\KwTo}$\text{\PLEN}_{o~p}$}{
    \eIf{$\text{\PARTITION}_{o~p~i} \geq 0$}{
      $\text{\UNSIGNED}_i \leftarrow \text{\PARTITION}_{o~p~i} \times 2$\;
    }{
    $\text{\UNSIGNED}_i \leftarrow (-\text{\PARTITION}_{o~p~i} - 1) \times 2 + 1$\;
    }
    $\text{\MSB}_i \leftarrow \lfloor \text{\UNSIGNED}_i \div 2 ^ \text{\RICE} \rfloor$\;
    $\text{\LSB}_i \leftarrow \text{\UNSIGNED}_i - (\text{\MSB}_i \times 2 ^ \text{\RICE})$\;
    $\text{\MSB}_i \rightarrow$ \WUNARY with stop bit 1\;
    $\text{\LSB}_i \rightarrow$ \WRITE $\text{\RICE}$ unsigned bits\;
  }
}
\Return encoded residual block\;
\EALGORITHM
}

\clearpage

\subsubsection{Encoding Partitions}
\label{flac:write_residual_partition}
{\relsize{-1}
\ALGORITHM{partition order $o$, predictor order, residual values, block size, maximum Rice parameter from encoding parameters}{Rice parameter, 1 or more residual partitions, total estimated size}
\SetKwData{ORDER}{predictor order}
\SetKwData{BLOCKSIZE}{block size}
\SetKwData{PSIZE}{partitions size}
\SetKwData{PLEN}{plength}
\SetKwData{PARTITION}{partition}
\SetKwData{RESIDUAL}{residual}
\SetKwData{PSUM}{partition sum}
\SetKwData{RICE}{Rice}
\SetKwData{MAXPARAMETER}{maximum Rice parameter}
\SetKw{BREAK}{break}
$\text{\PSIZE} \leftarrow 0$\;
\BlankLine
\For(\tcc*[f]{split residuals into partitions}){$p \leftarrow 0$ \emph{\KwTo}$2 ^ {o}$}{
  \eIf{$p = 0$}{
    $\text{\PLEN}_{0} \leftarrow \BLOCKSIZE \div 2 ^ {o} - \ORDER$\;
  }{
    $\text{\PLEN}_{p} \leftarrow \BLOCKSIZE \div 2 ^ {o}$\;
  }
  $\text{\PARTITION}_{p} \leftarrow$ get next $\text{\PLEN}_{p}$ values from \RESIDUAL\;
  \BlankLine
  $\text{\PSUM}_{p} \leftarrow \overset{\text{\PLEN}_{p} - 1}{\underset{i = 0}{\sum}} |\text{\PARTITION}_{p~i}|$\;
  \BlankLine
  $\text{\RICE}_{p} \leftarrow 0$\tcc*[r]{compute best Rice parameter for partition}
  \While{$\text{\PLEN}_{p} \times 2 ^ {\text{\RICE}_{p}} < \text{\PSUM}_{p}$}{
    \eIf{$\text{\RICE}_{p} < \MAXPARAMETER$}{
      $\text{\RICE}_{p} \leftarrow \text{\RICE}_{p} + 1$\;
    }{
      \BREAK\;
    }
  }
  \BlankLine
  \eIf(\tcc*[f]{add estimated size of partition to total size}){$\text{\RICE}_{p} > 0$}{
    $\text{\PSIZE} \leftarrow \text{\PSIZE} + 4 + ((1 + \text{\RICE}_{p}) \times \text{\PLEN}_{p}) + \left\lfloor\frac{\text{\PSUM}_{p}}{2 ^ {\text{\RICE}_{p} - 1}}\right\rfloor - \left\lfloor\frac{\text{\PLEN}_{p}}{2}\right\rfloor$\;
  }{
    $\text{\PSIZE} \leftarrow \text{\PSIZE} + 4 + \text{\PLEN}_{p} + (\text{\PSUM}_{p} \times 2) - \left\lfloor\frac{\text{\PLEN}_{p}}{2}\right\rfloor$\;
  }
}
\BlankLine
\Return $\left\lbrace\begin{tabular}{l}
$\text{\RICE}$ \\
$\text{\PARTITION}$ \\
$\text{\PSIZE}$ \\
\end{tabular}\right.$\;
\EALGORITHM
}

\begin{figure}[h]
\includegraphics{flac/figures/residual.pdf}
\end{figure}

\clearpage

\subsubsection{Residual Encoding Example}
Given a set of residuals \texttt{[2, 6, -2, 0, -1, -2, 3, -1, -3]},
block size of 10 and predictor order of 1:
{\relsize{-1}
  \begin{align*}
  \intertext{$\text{partition order}~o = 0$:}
  \textsf{plength}_{0~0} &\leftarrow 10 \div 2 ^ 0 - 1 = 9 \\
  \textsf{partition}_{0~0} &\leftarrow \texttt{[2, 6, -2, 0, -1, -2, 3, -1, -3]} \\
  \textsf{partition sum}_{0~0} &\leftarrow 2 + 6 + 2 + 0 + 1 + 2 + 3 + 1 + 3 = 20 \\
  \textsf{Rice}_{0~0} &\leftarrow \textbf{1}~~(9 \times 2 ^ \textbf{1} < 20 \text{ and } 9 \times 2 ^ \textbf{2} > 20) \\
  \textsf{partitions size}_0 &\leftarrow 0 + 4 + ((1 + 1) \times 9) + \left\lfloor\frac{20}{2 ^ 1 - 1}\right\rfloor - \left\lfloor\frac{9}{2}\right\rfloor = \textbf{38} \\
  \intertext{$\text{partition order}~o = 1$:}
  \textsf{plength}_{1~0} &\leftarrow 10 \div 2 ^ 1 - 1 = 4 \\
  \textsf{partition}_{1~0} &\leftarrow \texttt{[2, 6, -2, 0]} \\
  \textsf{partition sum}_{1~0} &\leftarrow 2 + 6 + 2 + 0 = 10 \\
  \textsf{Rice}_{1~0} &\leftarrow \textbf{1}~~(4 \times 2 ^ \textbf{1} < 10 \text{ and } 4 \times 2 ^ \textbf{2} > 10) \\
  \textsf{partitions size}_1 &\leftarrow 0 + 4 + ((1 + 1) \times 4) + \left\lfloor\frac{10}{2 ^ 1 - 1}\right\rfloor - \left\lfloor\frac{4}{2}\right\rfloor = \textbf{20} \\
  \textsf{plength}_{1~1} &\leftarrow 10 \div 2 ^ 1 = 5 \\
  \textsf{partition}_{1~1} &\leftarrow \texttt{[-1, -2, 3, -1, -3]} \\
  \textsf{partition sum}_{1~1} &\leftarrow 1 + 2 + 3 + 1 + 3 = 10 \\
  \textsf{Rice}_{1~1} &\leftarrow \textbf{0}~~(5 \times 2 ^ \textbf{0} < 10 \text{ and } 5 \times 2 ^ \textbf{1} = 10) \\
  \textsf{partitions size}_1 &\leftarrow \textbf{20} + 4 + 5 + (10 \times
  2) - \left\lfloor\frac{5}{2}\right\rfloor = \textbf{47}
\end{align*}}
\par
\noindent
Since block size of $10 \bmod 2 ^ 2 \neq 0$, we stop at partition order 1
because the list of residuals can't be divided equally into more partitions.
And because $\textsf{partitions size}_0$ of 38 is smaller than
$\textsf{partitions size}_1$ of 47, we use partition order 0
to encode our residuals into a single partition with 9 residuals.

\begin{figure}[h]
  \includegraphics{flac/figures/residuals-enc-example.pdf}
\end{figure}


\clearpage

%This work is licensed under the
%Creative Commons Attribution-Share Alike 3.0 United States License.
%To view a copy of this license, visit
%http://creativecommons.org/licenses/by-sa/3.0/us/ or send a letter to
%Creative Commons,
%171 Second Street, Suite 300,
%San Francisco, California, 94105, USA.

\subsection{Computing QLP Coefficients and Residual}
\label{alac:compute_qlp_coeffs}
{\relsize{-2}
\ALGORITHM{a list of signed PCM samples, sample size, encoding parameters}{a list of 4 or 8 signed QLP coefficients, a block of residual data; or a \textit{residual overflow} exception}
\SetKwData{SAMPLES}{subframe samples}
\SetKwData{WINDOWED}{windowed}
\SetKwData{AUTOCORRELATION}{autocorrelated}
\SetKwData{LPCOEFF}{LP coefficient}
\SetKwData{QLPCOEFF}{QLP coefficient}
\SetKwData{SAMPLESIZE}{sample size}
\SetKwData{RESIDUAL}{residual}
\SetKwData{RESIDUALBLOCK}{residual block}
$\WINDOWED \leftarrow$ \hyperref[alac:window]{window signed integer \SAMPLES}\;
$\AUTOCORRELATION \leftarrow$ \hyperref[alac:autocorrelate]{autocorrelate \WINDOWED}\;
\eIf{$\text{\AUTOCORRELATION}_0 \neq 0.0$}{
  $\LPCOEFF \leftarrow$ \hyperref[alac:compute_lp_coeffs]{compute LP coefficients from \AUTOCORRELATION}\;
  $\text{\QLPCOEFF}_3 \leftarrow$ \hyperref[alac:quantize_lp_coeffs]{quantize $\text{\LPCOEFF}_3$ at order 4}\;
  $\text{\QLPCOEFF}_7 \leftarrow$ \hyperref[alac:quantize_lp_coeffs]{quantize $\text{\LPCOEFF}_7$ at order 8}\;
  $\text{\RESIDUAL}_3 \leftarrow$ \hyperref[alac:calculate_residuals]{calculate residuals from $\text{\QLPCOEFF}_3$ and \SAMPLES}\;
  $\text{\RESIDUAL}_7 \leftarrow$ \hyperref[alac:calculate_residuals]{calculate residuals from $\text{\QLPCOEFF}_7$ and \SAMPLES}\;
  $\text{\RESIDUALBLOCK}_3 \leftarrow$ \hyperref[alac:write_residuals]{encode residual block from $\text{\RESIDUAL}_3$ with \SAMPLESIZE}\;
  $\text{\RESIDUALBLOCK}_7 \leftarrow$ \hyperref[alac:write_residuals]{encode residual block from $\text{\RESIDUAL}_7$ with \SAMPLESIZE}\;
  \eIf{$\LEN(\text{\RESIDUALBLOCK}_3) < (\LEN(\text{\RESIDUALBLOCK}_7) + 64~bits)$}{
    \Return $\left\lbrace\begin{tabular}{l}
    $\text{\QLPCOEFF}_3$ \\
    $\text{\RESIDUALBLOCK}_3$ \\
    \end{tabular}\right.$\;
  }{
    \Return $\left\lbrace\begin{tabular}{l}
    $\text{\QLPCOEFF}_7$ \\
    $\text{\RESIDUALBLOCK}_7$ \\
    \end{tabular}\right.$\;
  }
}(\tcc*[f]{all samples are 0}){
  \QLPCOEFF $\leftarrow$ \texttt{[0, 0, 0, 0]}\;
  $\text{\RESIDUAL} \leftarrow$ \hyperref[alac:calculate_residuals]{calculate residuals from $\text{\QLPCOEFF}$ and \SAMPLES}\;
  $\text{\RESIDUALBLOCK} \leftarrow$ \hyperref[alac:write_residuals]{encode residual block from $\text{\RESIDUAL}$ with \SAMPLESIZE}\;
  \Return $\left\lbrace\begin{tabular}{l}
    $\text{\QLPCOEFF}$ \\
    $\text{\RESIDUALBLOCK}$ \\
  \end{tabular}\right.$\;
}
\EALGORITHM
}

\subsubsection{Windowing the Input Samples}
\label{alac:window}
{\relsize{-1}
\ALGORITHM{a list of signed input sample integers}{a list of signed windowed samples as floats}
\SetKwFunction{TUKEY}{tukey}
\SetKwData{SAMPLECOUNT}{sample count}
\SetKwData{WINDOWED}{windowed}
\SetKwData{SAMPLE}{sample}
\For{$i \leftarrow 0$ \emph{\KwTo}\SAMPLECOUNT}{
  $\text{\WINDOWED}_i = \text{\SAMPLE}_i \times \TUKEY(i)$\;
}
\Return \WINDOWED\;
\EALGORITHM
\par
\noindent
where the \VAR{Tukey} function is defined as:
\begin{equation*}
tukey(n) =
\begin{cases}
\frac{1}{2} \times \left[1 + cos\left(\pi \times \left(\frac{2 \times n}{\alpha \times (N - 1)} - 1 \right)\right)\right] & \text{ if } 0 \leq n \leq \frac{\alpha \times (N - 1)}{2} \\
1 & \text{ if } \frac{\alpha \times (N - 1)}{2} \leq n \leq (N - 1) \times (1 - \frac{\alpha}{2}) \\
\frac{1}{2} \times \left[1 + cos\left(\pi \times \left(\frac{2 \times n}{\alpha \times (N - 1)} - \frac{2}{\alpha} + 1 \right)\right)\right] & \text{ if } (N - 1) \times (1 - \frac{\alpha}{2}) \leq n \leq (N - 1) \\
\end{cases}
\end{equation*}
\par
\noindent
$N$ is the total number of input samples and $\alpha$ is $\nicefrac{1}{2}$.
\par
\noindent
{\relsize{-2}
\begin{tabular}{r|rcrcr}
$i$ & $\textsf{sample}_i$ & & \texttt{tukey}($i$) & & $\textsf{windowed}_i$ \\
\hline
0 & \texttt{0} & $\times$ & \texttt{0.00} & = & \texttt{0.00} \\
1 & \texttt{16} & $\times$ & \texttt{0.19} & = & \texttt{3.01} \\
2 & \texttt{31} & $\times$ & \texttt{0.61} & = & \texttt{18.95} \\
3 & \texttt{44} & $\times$ & \texttt{0.95} & = & \texttt{41.82} \\
4 & \texttt{54} & $\times$ & \texttt{1.00} & = & \texttt{54.00} \\
5 & \texttt{61} & $\times$ & \texttt{1.00} & = & \texttt{61.00} \\
6 & \texttt{64} & $\times$ & \texttt{1.00} & = & \texttt{64.00} \\
7 & \texttt{63} & $\times$ & \texttt{1.00} & = & \texttt{63.00} \\
8 & \texttt{58} & $\times$ & \texttt{1.00} & = & \texttt{58.00} \\
9 & \texttt{49} & $\times$ & \texttt{1.00} & = & \texttt{49.00} \\
10 & \texttt{38} & $\times$ & \texttt{1.00} & = & \texttt{38.00} \\
11 & \texttt{24} & $\times$ & \texttt{0.95} & = & \texttt{22.81} \\
12 & \texttt{8} & $\times$ & \texttt{0.61} & = & \texttt{4.89} \\
13 & \texttt{-8} & $\times$ & \texttt{0.19} & = & \texttt{-1.51} \\
14 & \texttt{-24} & $\times$ & \texttt{0.00} & = & \texttt{0.00} \\
\end{tabular}
}
}

\clearpage

\subsubsection{Autocorrelating Windowed Samples}
\label{alac:autocorrelate}
{\relsize{-1}
\ALGORITHM{a list of signed windowed samples}{a list of signed autocorrelation values}
\SetKwData{LAG}{lag}
\SetKwData{AUTOCORRELATION}{autocorrelated}
\SetKwData{TOTALSAMPLES}{total samples}
\SetKwData{WINDOWED}{windowed}
\For{$\LAG \leftarrow 0$ \emph{\KwTo}9}{
  $\text{\AUTOCORRELATION}_{\text{\LAG}} = \overset{\text{\TOTALSAMPLES} - \text{\LAG} - 1}{\underset{i = 0}{\sum}}\text{\WINDOWED}_i \times \text{\WINDOWED}_{i + \text{\LAG}}$\;
}
\Return \AUTOCORRELATION\;
\EALGORITHM
}

\subsubsection{Autocorrelation Example}
{\relsize{-1}
\begin{multicols}{2}
\begin{tabular}{rrrrr}
  \texttt{0.00} & $\times$ & \texttt{0.00} & $=$ & \texttt{0.00} \\
  \texttt{3.01} & $\times$ & \texttt{3.01} & $=$ & \texttt{9.07} \\
  \texttt{18.95} & $\times$ & \texttt{18.95} & $=$ & \texttt{359.07} \\
  \texttt{41.82} & $\times$ & \texttt{41.82} & $=$ & \texttt{1749.02} \\
  \texttt{54.00} & $\times$ & \texttt{54.00} & $=$ & \texttt{2916.00} \\
  \texttt{61.00} & $\times$ & \texttt{61.00} & $=$ & \texttt{3721.00} \\
  \texttt{64.00} & $\times$ & \texttt{64.00} & $=$ & \texttt{4096.00} \\
  \texttt{63.00} & $\times$ & \texttt{63.00} & $=$ & \texttt{3969.00} \\
  \texttt{58.00} & $\times$ & \texttt{58.00} & $=$ & \texttt{3364.00} \\
  \texttt{49.00} & $\times$ & \texttt{49.00} & $=$ & \texttt{2401.00} \\
  \texttt{38.00} & $\times$ & \texttt{38.00} & $=$ & \texttt{1444.00} \\
  \texttt{22.81} & $\times$ & \texttt{22.81} & $=$ & \texttt{520.37} \\
  \texttt{4.89} & $\times$ & \texttt{4.89} & $=$ & \texttt{23.91} \\
  \texttt{-1.51} & $\times$ & \texttt{-1.51} & $=$ & \texttt{2.27} \\
  \texttt{0.00} & $\times$ & \texttt{0.00} & $=$ & \texttt{0.00} \\
  \hline
  \multicolumn{3}{r}{$\textsf{autocorrelation}_0$} & $=$ & \texttt{24574.71} \\
\end{tabular}
\par
\begin{tabular}{rrrrr}
  \texttt{0.00} & $\times$ & \texttt{3.01} & $=$ & \texttt{0.00} \\
  \texttt{3.01} & $\times$ & \texttt{18.95} & $=$ & \texttt{57.08} \\
  \texttt{18.95} & $\times$ & \texttt{41.82} & $=$ & \texttt{792.48} \\
  \texttt{41.82} & $\times$ & \texttt{54.00} & $=$ & \texttt{2258.35} \\
  \texttt{54.00} & $\times$ & \texttt{61.00} & $=$ & \texttt{3294.00} \\
  \texttt{61.00} & $\times$ & \texttt{64.00} & $=$ & \texttt{3904.00} \\
  \texttt{64.00} & $\times$ & \texttt{63.00} & $=$ & \texttt{4032.00} \\
  \texttt{63.00} & $\times$ & \texttt{58.00} & $=$ & \texttt{3654.00} \\
  \texttt{58.00} & $\times$ & \texttt{49.00} & $=$ & \texttt{2842.00} \\
  \texttt{49.00} & $\times$ & \texttt{38.00} & $=$ & \texttt{1862.00} \\
  \texttt{38.00} & $\times$ & \texttt{22.81} & $=$ & \texttt{866.84} \\
  \texttt{22.81} & $\times$ & \texttt{4.89} & $=$ & \texttt{111.55} \\
  \texttt{4.89} & $\times$ & \texttt{-1.51} & $=$ & \texttt{-7.36} \\
  \texttt{-1.51} & $\times$ & \texttt{0.00} & $=$ & \texttt{0.00} \\
  \hline
  \multicolumn{3}{r}{$\textsf{autocorrelation}_1$} & $=$ & \texttt{23666.93} \\
\end{tabular}
\par
\begin{tabular}{rrrrr}
  \texttt{0.00} & $\times$ & \texttt{18.95} & $=$ & \texttt{0.00} \\
  \texttt{3.01} & $\times$ & \texttt{41.82} & $=$ & \texttt{125.97} \\
  \texttt{18.95} & $\times$ & \texttt{54.00} & $=$ & \texttt{1023.25} \\
  \texttt{41.82} & $\times$ & \texttt{61.00} & $=$ & \texttt{2551.10} \\
  \texttt{54.00} & $\times$ & \texttt{64.00} & $=$ & \texttt{3456.00} \\
  \texttt{61.00} & $\times$ & \texttt{63.00} & $=$ & \texttt{3843.00} \\
  \texttt{64.00} & $\times$ & \texttt{58.00} & $=$ & \texttt{3712.00} \\
  \texttt{63.00} & $\times$ & \texttt{49.00} & $=$ & \texttt{3087.00} \\
  \texttt{58.00} & $\times$ & \texttt{38.00} & $=$ & \texttt{2204.00} \\
  \texttt{49.00} & $\times$ & \texttt{22.81} & $=$ & \texttt{1117.77} \\
  \texttt{38.00} & $\times$ & \texttt{4.89} & $=$ & \texttt{185.82} \\
  \texttt{22.81} & $\times$ & \texttt{-1.51} & $=$ & \texttt{-34.36} \\
  \texttt{4.89} & $\times$ & \texttt{0.00} & $=$ & \texttt{0.00} \\
  \hline
  \multicolumn{3}{r}{$\textsf{autocorrelation}_2$} & $=$ & \texttt{21271.56} \\
\end{tabular}
\par
\begin{tabular}{rrrrr}
  \texttt{0.00} & $\times$ & \texttt{41.82} & $=$ & \texttt{0.00} \\
  \texttt{3.01} & $\times$ & \texttt{54.00} & $=$ & \texttt{162.65} \\
  \texttt{18.95} & $\times$ & \texttt{61.00} & $=$ & \texttt{1155.89} \\
  \texttt{41.82} & $\times$ & \texttt{64.00} & $=$ & \texttt{2676.56} \\
  \texttt{54.00} & $\times$ & \texttt{63.00} & $=$ & \texttt{3402.00} \\
  \texttt{61.00} & $\times$ & \texttt{58.00} & $=$ & \texttt{3538.00} \\
  \texttt{64.00} & $\times$ & \texttt{49.00} & $=$ & \texttt{3136.00} \\
  \texttt{63.00} & $\times$ & \texttt{38.00} & $=$ & \texttt{2394.00} \\
  \texttt{58.00} & $\times$ & \texttt{22.81} & $=$ & \texttt{1323.07} \\
  \texttt{49.00} & $\times$ & \texttt{4.89} & $=$ & \texttt{239.61} \\
  \texttt{38.00} & $\times$ & \texttt{-1.51} & $=$ & \texttt{-57.23} \\
  \texttt{22.81} & $\times$ & \texttt{0.00} & $=$ & \texttt{0.00} \\
  \hline
  \multicolumn{3}{r}{$\textsf{autocorrelation}_3$} & $=$ & \texttt{17970.57} \\
\end{tabular}
\par
\begin{tabular}{rrrrr}
  \texttt{0.00} & $\times$ & \texttt{54.00} & $=$ & \texttt{0.00} \\
  \texttt{3.01} & $\times$ & \texttt{61.00} & $=$ & \texttt{183.74} \\
  \texttt{18.95} & $\times$ & \texttt{64.00} & $=$ & \texttt{1212.74} \\
  \texttt{41.82} & $\times$ & \texttt{63.00} & $=$ & \texttt{2634.74} \\
  \texttt{54.00} & $\times$ & \texttt{58.00} & $=$ & \texttt{3132.00} \\
  \texttt{61.00} & $\times$ & \texttt{49.00} & $=$ & \texttt{2989.00} \\
  \texttt{64.00} & $\times$ & \texttt{38.00} & $=$ & \texttt{2432.00} \\
  \texttt{63.00} & $\times$ & \texttt{22.81} & $=$ & \texttt{1437.13} \\
  \texttt{58.00} & $\times$ & \texttt{4.89} & $=$ & \texttt{283.62} \\
  \texttt{49.00} & $\times$ & \texttt{-1.51} & $=$ & \texttt{-73.80} \\
  \texttt{38.00} & $\times$ & \texttt{0.00} & $=$ & \texttt{0.00} \\
  \hline
  \multicolumn{3}{r}{$\textsf{autocorrelation}_4$} & $=$ & \texttt{14231.18} \\
\end{tabular}
\end{multicols}
}

\clearpage

\subsubsection{LP Coefficient Calculation}
\label{alac:compute_lp_coeffs}
{\relsize{-1}
\ALGORITHM{a list of autocorrelation floats}{a list of LP coefficient lists}
\SetKwData{LPCOEFF}{LP coefficient}
\SetKwData{ERROR}{error}
\SetKwData{AUTOCORRELATION}{autocorrelation}
\begin{tabular}{rcl}
$\kappa_0$ &$\leftarrow$ & $ \AUTOCORRELATION_1 \div \AUTOCORRELATION_0$ \\
$\LPCOEFF_{0~0}$ &$\leftarrow$ & $ \kappa_0$ \\
$\ERROR_0$ &$\leftarrow$ & $ \AUTOCORRELATION_0 \times (1 - {\kappa_0} ^ 2)$ \\
\end{tabular}\;
\For{$i \leftarrow 1$ \emph{\KwTo}8}{
  \tcc{"zip" all of the previous row's LP coefficients
    \newline
    and the reversed autocorrelation values from 1 to i + 1
    \newline
    into ($c$,$a$) pairs
    \newline
    $q_i$ is $\AUTOCORRELATION_{i + 1}$ minus the sum of those multiplied ($c$,$a$) pairs}
  $q_i \leftarrow \AUTOCORRELATION_{i + 1}$\;
  \For{$j \leftarrow 0$ \emph{\KwTo}i}{
    $q_i \leftarrow q_i - (\LPCOEFF_{(i - 1)~j} \times \AUTOCORRELATION_{i - j})$\;
  }
  \BlankLine
  \tcc{"zip" all of the previous row's LP coefficients
    \newline
    and the previous row's LP coefficients reversed
    \newline
    into ($c$,$r$) pairs}
  $\kappa_i = q_i \div \ERROR_{i - 1}$\;
  \For{$j \leftarrow 0$ \emph{\KwTo}i}{
    \tcc{then build a new coefficient list of $c - (\kappa_i * r)$ for each ($c$,$r$) pair}
    $\LPCOEFF_{i~j} \leftarrow \LPCOEFF_{(i - 1)~j} - (\kappa_i \times \LPCOEFF_{(i - 1)~(i - j - 1)})$\;
  }
  $\text{\LPCOEFF}_{i~i} \leftarrow \kappa_i$\tcc*[r]{and append $\kappa_i$ as the final coefficient in that list}
  \BlankLine
  $\ERROR_i \leftarrow \ERROR_{i - 1} \times (1 - {\kappa_i}^2)$\;
}
\Return \LPCOEFF\;
\EALGORITHM
}

\begin{landscape}

\subsubsection{LP Coefficient Calculation Example}
\begin{table}[h]
{\relsize{-1}
\begin{tabular}{r|r}
$i$ & $\textsf{autocorrelation}_i$ \\
\hline
0 & \texttt{24598.25} \\
1 & \texttt{23694.34} \\
2 & \texttt{21304.57} \\
3 & \texttt{18007.86} \\
4 & \texttt{14270.30} \\
\end{tabular}
}
\end{table}

\begin{table}[h]
{\relsize{-1}
\renewcommand{\arraystretch}{1.45}
\begin{tabular}{|>{$}r<{$}||>{$}r<{$}|>{$}r<{$}|>{$}r<{$}|>{$}r<{$}|}
\hline
k_0 &
\multicolumn{4}{>{$}l<{$}|}{\texttt{23694.34} \div \texttt{24598.25} = \texttt{0.96}} \\
\textsf{LP coefficient}_{0~0} & \texttt{\color{blue}0.96} & & & \\
\textsf{error}_0 &
\multicolumn{4}{>{$}l<{$}|}{\texttt{24598.25} \times (1 - \texttt{0.96} ^ 2) = \texttt{1774.62}} \\
\hline
q_1 & \multicolumn{4}{>{$}l<{$}|}{\texttt{21304.57} - (\texttt{0.96} \times \texttt{23694.34}) = \texttt{-1519.07}} \\
k_1 & \multicolumn{4}{>{$}l<{$}|}{\texttt{-1519.07} \div \texttt{1774.62} = \texttt{-0.86}} \\
\textsf{LP coefficient}_{1~i} &
\texttt{0.96} -(\texttt{-0.86} \times \texttt{0.96}) = \texttt{\color{blue}1.79} &
\texttt{\color{blue}-0.86} & & \\
\textsf{error}_1 & \multicolumn{4}{>{$}l<{$}|}{\texttt{1774.62} \times (1 - \texttt{-0.86} ^ 2) = \texttt{474.30}} \\
\hline
q_2 & \multicolumn{4}{>{$}l<{$}|}{\texttt{18007.86} - (\texttt{1.79} \times \texttt{21304.57} + \texttt{-0.86} \times \texttt{23694.34}) = \texttt{201.96}} \\
k_2 & \multicolumn{4}{>{$}l<{$}|}{\texttt{201.96} \div \texttt{474.30} = \texttt{0.43}} \\
\textsf{LP coefficient}_{2~i} &
\texttt{1.79} -(\texttt{0.43} \times \texttt{-0.86}) = \texttt{\color{blue}2.15} &
\texttt{-0.86} -(\texttt{0.43} \times \texttt{1.79}) = \texttt{\color{blue}-1.62} &
\texttt{\color{blue}0.43} & \\
\textsf{error}_2 & \multicolumn{4}{>{$}l<{$}|}{\texttt{474.30} \times (1 - \texttt{0.43} ^ 2) = \texttt{388.31}} \\
\hline
q_3 & \multicolumn{4}{>{$}l<{$}|}{\texttt{14270.30} - (\texttt{2.15} \times \texttt{18007.86} + \texttt{-1.62} \times \texttt{21304.57} + \texttt{0.43} \times \texttt{23694.34}) = \texttt{-122.06}} \\
k_3 & \multicolumn{4}{>{$}l<{$}|}{\texttt{-122.06} \div \texttt{388.31} = \texttt{-0.31}} \\
\textsf{LP coefficient}_{3~i} &
\texttt{2.15} -(\texttt{-0.31} \times \texttt{0.43}) = \texttt{\color{blue}2.29} &
\texttt{-1.62} -(\texttt{-0.31} \times \texttt{-1.62}) = \texttt{\color{blue}-2.13} &
\texttt{0.43} -(\texttt{-0.31} \times \texttt{2.15}) = \texttt{\color{blue}1.10} &
\texttt{\color{blue}-0.31} \\
\textsf{error}_3 & \multicolumn{4}{>{$}l<{$}|}{\texttt{388.31} \times (1 - \texttt{-0.31} ^ 2) = \texttt{349.94}} \\
\hline
\end{tabular}
\renewcommand{\arraystretch}{1.0}
}
\end{table}

\end{landscape}

\subsubsection{Quantizing LP Coefficients}
\label{alac:quantize_lp_coeffs}
\ALGORITHM{LP coefficients, an order value of 4 or 8}{QLP coefficients as a list of signed integers}
\SetKwData{ORDER}{order}
\SetKwFunction{MIN}{min}
\SetKwFunction{MAX}{max}
\SetKwFunction{ROUND}{round}
\SetKwData{QLPMIN}{QLP min}
\SetKwData{QLPMAX}{QLP max}
\SetKwData{LPCOEFF}{LP coefficient}
\SetKwData{QLPCOEFF}{QLP coefficient}
\tcc{QLP min and max are the smallest and largest QLP coefficients that fit in a signed field that's 16 bits wide}
$\QLPMIN \leftarrow 2 ^ \text{15} - 1$\;
$\QLPMAX \leftarrow -(2 ^ \text{15})$\;
$e \leftarrow 0.0$\;
\For{$i \leftarrow 0$ \emph{\KwTo}\ORDER}{
  $e \leftarrow e + \text{\LPCOEFF}_{\ORDER - 1~i} \times 2 ^ 9$\;
  $\text{\QLPCOEFF}_i \leftarrow \MIN(\MAX(\ROUND(e)~,~\text{\QLPMIN})~,~\text{\QLPMAX})$\;
  $e \leftarrow e - \text{\QLPCOEFF}_i$\;
}
\Return \QLPCOEFF\;
\EALGORITHM

\clearpage

\subsubsection{Quantizing Coefficients Example}
\begin{align*}
e &\leftarrow \texttt{0.00} + \texttt{2.29} \times 2 ^ 9 = \texttt{1170.49} \\
\textsf{QLP coefficient}_0 &\leftarrow \texttt{round}(\texttt{1170.49}) = \texttt{\color{blue}1170} \\
e &\leftarrow \texttt{1170.49} - 1170 = \texttt{0.49} \\
e &\leftarrow \texttt{0.49} + \texttt{-2.13} \times 2 ^ 9 = \texttt{-1087.81} \\
\textsf{QLP coefficient}_1 &\leftarrow \texttt{round}(\texttt{-1087.81}) = \texttt{\color{blue}-1088} \\
e &\leftarrow \texttt{-1087.81} - -1088 = \texttt{0.19} \\
e &\leftarrow \texttt{0.19} + \texttt{1.10} \times 2 ^ 9 = \texttt{564.59} \\
\textsf{QLP coefficient}_2 &\leftarrow\texttt{round}(\texttt{564.59}) = \texttt{\color{blue}565} \\
e &\leftarrow \texttt{564.59} - 565 = \texttt{-0.41} \\
e &\leftarrow \texttt{-0.41} + \texttt{-0.31} \times 2 ^ 9 = \texttt{-161.35} \\
\textsf{QLP coefficient}_3 &\leftarrow \texttt{round}(\texttt{-161.35}) = \texttt{\color{blue}-161} \\
e &\leftarrow \texttt{-161.35} - -161 = \texttt{-0.35} \\
\end{align*}


\clearpage

\subsection{Calculating Frame CRC-16}
\label{flac:calculate_crc16}
CRC-16 is used to checksum the entire FLAC frame, including the header
and any padding bits after the final subframe.
Given a byte of input and the previous CRC-16 checksum,
or 0 as an initial value, the current checksum can be calculated as follows:
\begin{equation}
\text{checksum}_i = \texttt{CRC16}(byte\xor(\text{checksum}_{i - 1} \gg 8 ))\xor(\text{checksum}_{i - 1} \ll 8)
\end{equation}
\par
\noindent
and the checksum is always truncated to 16-bits.
\begin{table}[h]
{\relsize{-3}\ttfamily
\begin{tabular}{|r||r|r|r|r|r|r|r|r|r|r|r|r|r|r|r|r|}
\hline
 & 0x?0 & 0x?1 & 0x?2 & 0x?3 & 0x?4 & 0x?5 & 0x?6 & 0x?7 & 0x?8 & 0x?9 & 0x?A & 0x?B & 0x?C & 0x?D & 0x?E & 0x?F \\
\hline
0x0? & 0000 & 8005 & 800f & 000a & 801b & 001e & 0014 & 8011 & 8033 & 0036 & 003c & 8039 & 0028 & 802d & 8027 & 0022 \\
0x1? & 8063 & 0066 & 006c & 8069 & 0078 & 807d & 8077 & 0072 & 0050 & 8055 & 805f & 005a & 804b & 004e & 0044 & 8041 \\
0x2? & 80c3 & 00c6 & 00cc & 80c9 & 00d8 & 80dd & 80d7 & 00d2 & 00f0 & 80f5 & 80ff & 00fa & 80eb & 00ee & 00e4 & 80e1 \\
0x3? & 00a0 & 80a5 & 80af & 00aa & 80bb & 00be & 00b4 & 80b1 & 8093 & 0096 & 009c & 8099 & 0088 & 808d & 8087 & 0082 \\
0x4? & 8183 & 0186 & 018c & 8189 & 0198 & 819d & 8197 & 0192 & 01b0 & 81b5 & 81bf & 01ba & 81ab & 01ae & 01a4 & 81a1 \\
0x5? & 01e0 & 81e5 & 81ef & 01ea & 81fb & 01fe & 01f4 & 81f1 & 81d3 & 01d6 & 01dc & 81d9 & 01c8 & 81cd & 81c7 & 01c2 \\
0x6? & 0140 & 8145 & 814f & 014a & 815b & 015e & 0154 & 8151 & 8173 & 0176 & 017c & 8179 & 0168 & 816d & 8167 & 0162 \\
0x7? & 8123 & 0126 & 012c & 8129 & 0138 & 813d & 8137 & 0132 & 0110 & 8115 & 811f & 011a & 810b & 010e & 0104 & 8101 \\
0x8? & 8303 & 0306 & 030c & 8309 & 0318 & 831d & 8317 & 0312 & 0330 & 8335 & 833f & 033a & 832b & 032e & 0324 & 8321 \\
0x9? & 0360 & 8365 & 836f & 036a & 837b & 037e & 0374 & 8371 & 8353 & 0356 & 035c & 8359 & 0348 & 834d & 8347 & 0342 \\
0xA? & 03c0 & 83c5 & 83cf & 03ca & 83db & 03de & 03d4 & 83d1 & 83f3 & 03f6 & 03fc & 83f9 & 03e8 & 83ed & 83e7 & 03e2 \\
0xB? & 83a3 & 03a6 & 03ac & 83a9 & 03b8 & 83bd & 83b7 & 03b2 & 0390 & 8395 & 839f & 039a & 838b & 038e & 0384 & 8381 \\
0xC? & 0280 & 8285 & 828f & 028a & 829b & 029e & 0294 & 8291 & 82b3 & 02b6 & 02bc & 82b9 & 02a8 & 82ad & 82a7 & 02a2 \\
0xD? & 82e3 & 02e6 & 02ec & 82e9 & 02f8 & 82fd & 82f7 & 02f2 & 02d0 & 82d5 & 82df & 02da & 82cb & 02ce & 02c4 & 82c1 \\
0xE? & 8243 & 0246 & 024c & 8249 & 0258 & 825d & 8257 & 0252 & 0270 & 8275 & 827f & 027a & 826b & 026e & 0264 & 8261 \\
0xF? & 0220 & 8225 & 822f & 022a & 823b & 023e & 0234 & 8231 & 8213 & 0216 & 021c & 8219 & 0208 & 820d & 8207 & 0202 \\
\hline
\end{tabular}
}
\end{table}
\par
\noindent
For example, given the frame bytes:
\texttt{FF F8 CC 1C 00 C0 EB 00 00 00 00 00 00 00 00},
the frame's CRC-16 can be calculated:
{\relsize{-2}
\begin{align*}
\CRCSIXTEEN{0}{0xFF}{0x0000}{0xFF}{0x0000}{0x0202} \\
\CRCSIXTEEN{1}{0xF8}{0x0202}{0xFA}{0x0200}{0x001C} \\
\CRCSIXTEEN{2}{0xCC}{0x001C}{0xCC}{0x1C00}{0x1EA8} \\
\CRCSIXTEEN{3}{0x1C}{0x1EA8}{0x02}{0xA800}{0x280F} \\
\CRCSIXTEEN{4}{0x00}{0x280F}{0x28}{0x0F00}{0x0FF0} \\
\CRCSIXTEEN{5}{0xC0}{0x0FF0}{0xCF}{0xF000}{0xF2A2} \\
\CRCSIXTEEN{6}{0xEB}{0xF2A2}{0x19}{0xA200}{0x2255} \\
\CRCSIXTEEN{7}{0x00}{0x2255}{0x22}{0x5500}{0x55CC} \\
\CRCSIXTEEN{8}{0x00}{0x55CC}{0x55}{0xCC00}{0xCDFE} \\
\CRCSIXTEEN{9}{0x00}{0xCDFE}{0xCD}{0xFE00}{0x7CAD} \\
\CRCSIXTEEN{10}{0x00}{0x7CAD}{0x7C}{0xAD00}{0x2C0B} \\
\CRCSIXTEEN{11}{0x00}{0x2C0B}{0x2C}{0x0B00}{0x8BEB} \\
\CRCSIXTEEN{12}{0x00}{0x8BEB}{0x8B}{0xEB00}{0xE83A} \\
\CRCSIXTEEN{13}{0x00}{0xE83A}{0xE8}{0x3A00}{0x3870} \\
\CRCSIXTEEN{14}{0x00}{0x3870}{0x38}{0x7000}{0xF093} \\
\intertext{Thus, the next two bytes after the final subframe should be
\texttt{0xF0} and \texttt{0x93}.
Again, when the checksum bytes are run through the checksumming procedure:}
\CRCSIXTEEN{15}{0xF0}{0xF093}{0x00}{0x9300}{0x9300} \\
\CRCSIXTEEN{16}{0x93}{0x9300}{0x00}{0x0000}{0x0000}
\end{align*}
the result will also always be 0, just as in the CRC-8.
}
